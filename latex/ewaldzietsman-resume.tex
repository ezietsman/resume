%%%%%%%%%%%%%%%%%%%%%%%%%%%%%%%%%%%%%%%%%
% Friggeri Resume/CV
% XeLaTeX Template
% Version 1.0 (5/5/13)
%
% This template has been downloaded from:
% http://www.LaTeXTemplates.com
%
% Original author:
% Adrien Friggeri (adrien@friggeri.net)
% https://github.com/afriggeri/CV
%
% License:
% CC BY-NC-SA 3.0 (http://creativecommons.org/licenses/by-nc-sa/3.0/)
%
% Important notes:
% This template needs to be compiled with XeLaTeX and the bibliography, if used,
% needs to be compiled with biber rather than bibtex.
%
%%%%%%%%%%%%%%%%%%%%%%%%%%%%%%%%%%%%%%%%%

\documentclass[]{friggeri-cv} % Add 'print' as an option into the square bracket to remove colors from this template for printing
\usepackage{multicol}
\usepackage[none]{hyphenat}
\addbibresource{ezietsman-ref.bib} % Specify the bibliography file to include publications

\newcommand{\mnras}{MNRAS} 
\newcommand{\apj}{Astrophysical Journal} 
\begin{document}

\header{ewald}{zietsman}{software developer} % Your name and current job title/field

%----------------------------------------------------------------------------------------
%	SIDEBAR SECTION
%----------------------------------------------------------------------------------------

\begin{aside} % In the aside, each new line forces a line break
\section{contact}
Observatory
Cape Town
South Africa
~
~
\textbf{email}:
\href{mailto:ewald.zietsman@gmail.com}{ewald.zietsman@gmail.com}
~
\textbf{website}:
\href{http://ezietsman.github.io/}{ezietsman.github.io}
\section{languages}
english 
afrikaans
\section{personal info}
age: 32
nationality: South African
ID: 8202085009086
%\section{programming languages}
%python
%javascript
%\LaTeX
%cython
%fortran
\end{aside}
%----------------------------------------------------------------------------------------
%	WORK EXPERIENCE SECTION
%----------------------------------------------------------------------------------------
\section{experience}
\begin{entrylist}
\entry
{2011 -- }
{\href{http://www.siyavula.com}{Siyavula Education}}
{Cape Town}
{%
\textbf{Technical Coordinator / Software Developer}
\begin{itemize}
\setlength{\itemsep}{5pt}
\item Development and maintenance of Python software for XML document validation and conversion to publishing formats including HTML5, \LaTeX, EPub 3.0 and Mobile Web.
\item Implementation and maintenance of \LaTeX~style sheets for print book PDF generation. See PDF downloads at:
    \subitem \url{http://www.thunderboltkids.co.za}
    \subitem \url{http://www.curious.org.za}
    \subitem \url{http://www.everythingscience.co.za}
    \subitem \url{http://www.everythingmaths.co.za}

\item Management of technical aspects of textbook translations including:
\begin{itemize}
    \renewcommand{\labelitemii}{$\cdot$}
    \item Maintenance of \emph{Transifex} server.
    \item XML conversion to and from translation formats.
\end{itemize}

\item Developing proof of concept models and testing of new tools and technologies for possible inclusion in current processes.

\item Development of custom modules in the Plone CMS related to online hosting of textbooks.

\item Contributing to related open source software development projects including:
\begin{itemize}
    \renewcommand{\labelitemii}{$\cdot$}
    \item OERPub Editor:~\url{http://oerpub.github.io/Aloha-Editor}
    \item AnnotatorJS:~\url{http://annotatorjs.org}
\end{itemize}
\end{itemize}
}
\end{entrylist}
\begin{entrylist}
%
\entry
{2008}
{\href{http://www.uct.ac.za}{University of Cape Town}}
{School of Geomatics}
{%
\textbf{Lecturer}
\begin{itemize}
\setlength{\itemsep}{5pt}
    \item Lectured spherical trigonometry and map projections as part of Geographical Informations Systems I (2nd year course).
    \item Lectured part of the Basic Surveying I course (2nd year course).
\end{itemize}
}
\end{entrylist}
\begin{entrylist}
%
\entry
{2006 -- 2010}
{\href{http://www.uct.ac.za}{University of Cape Town}}
{Department of Astronomy}
{%
\textbf{Course Tutor}

Astronomy I, Astronomy II, Spectroscopy NASSP Honours Course.
}
\end{entrylist}
\begin{entrylist}
%
\entry
{2001 -- 2004}
{\textcolor{maroonsblue}{James Mahon Landmeters}}
{Pretoria}
{%
\textbf{Student Land Surveyor}

Assisted with various types of survey projects including precise engineering surveys, sectional title surveys, boundary disputes, setting out of townships and parcels, topographical surveys. Also assisted with other cadastral type work such as subdivisions of farms and parcels, servitude surveys and beacon relocation surveys. 
}
\end{entrylist}
\pagebreak
\newgeometry{left=2.5cm, right=2.5cm}
\renewcommand{\entry}[4]{%
  #1&\parbox[t]{13.5cm}{%
    \textbf{#2}%
    \hfill%
    {\footnotesize\addfontfeature{Color=lightgray} #3}\\%
    #4\vspace{3\parsep}%
  }\\}

%----------------------------------------------------------------------------------------
%	EDUCATION SECTION
%----------------------------------------------------------------------------------------

\section{education}
\begin{entrylist}
%------------------------------------------------
\entry
{2009 -- 2011 \\(Incomplete)}
{\textbf D.Phil. in Astronomy}
{University of South Africa}
{{\textbf{Thesis:} A Study of Selected Magnetic Cataclysmic Variables}

{\textbf{Relevant skills:}
    \begin{itemize}
        \item Reduction and analysis of high time-resolution observations obtained using the Southern African Large Telescope (SALT).
        \item Computationally intensive calculations (genetic algorithms, utilising neural networks) performed
        using ROCKS (\url{www.rocksclusters.org}) cluster computing environment.
        \item Scientific observations using the High-speed polarimeter (HIPPO) on the SAAO 1.9m telescope in
        Sutherland.
        \item Development of custom analysis and visualisation software using Python and related tools.
\end{itemize}}
}
\end{entrylist}
\begin{entrylist}
%------------------------------------------------
%------------------------------------------------
\entry
{2006 -- 2008}
{\textbf M.Sc. in Astrophysics and Space Science}
{University of Cape Town}
{{\textbf{Coursework:} Plasma Physics, Magnetohydrodynamics, Extragalactic Astronomy, Observational
Cosmology and Cataclysmic Variables.}

{\textbf{Dissertation:} High-speed Photometry and Spectroscopy of the Cataclysmic Variable EC2117-54: Exploring New Avenues With the Southern African Large Telescope.}}
\end{entrylist}
\begin{entrylist}
%------------------------------------------------
\entry
{2005}
{\textbf B.Sc.(Hons) in Astrophysics and Space Science}
{University of Cape Town}
{{\textbf{Coursework:} Electrodynamics, Quantum Mechanics, Stellar Atmospheres, Stellar Structure and Evolution, Computational Physics, General Relativity, Galaxies, Observational Techniques and Radio Astronomy.}}
%------------------------------------------------
\entry
{2001 -- 2004}
{\textbf B.Sc. in Geomatics}
{University of Cape Town}
{{\textbf{Coursework:} Surveying, Geographic Information Systems, Computer Science, Remote Sensing,
Mathematics, Physics, Astronomy, Numerical Methods, Land Law, Engineering Surveying and Photogrammetry.}}
%------------------------------------------------
%------------------------------------------------
\end{entrylist}

%----------------------------------------------------------------------------------------
%	SKILLS EXPERIENCE SECTION
%----------------------------------------------------------------------------------------

\section{skills}

\textbf{Programming languages:} Python, Javascript, Fortran, C/C++, \LaTeX.

\textbf{Operating systems:} Linux (Ubuntu, Fedora Core, Mandriva), Microsoft Windows.

\textbf{Applications:} Git, Pyramid, Plone, Django, scipy, matplotlib, pyfits, numpy, vpython, cython, swig, f2py, pygame, parallelpython, Subversion, Mercurial, IRAF, \LaTeX.

\textbf{Surveying instruments:} Digital and Opto-mechanical Theodolites, Static and Real Time Kinematic GPS
    systems, Precise Levels, Dumpy Levels.
    
\textbf{Astronomical observing:} 110 nights observing experience on SAAO 1.9m and SAAO 1.0m Telescopes using
     the SAAO High-speed Polarimeter, UCT CCD Photometer, SAAO CCD Photometer.

\textbf{Miscellaneous:} exceptional analytical and problem solving skills, technical software programming and design capabilities, object-oriented programming, strong verbal and written communication skills.



\begin{entrylist}
\end{entrylist}
\section{publications}
\printbibsection{article}{articles} % Print all articles from the bibliography
\printbibsection{inproceedings}{conference proceedings} % Print all articles from the bibliography

%----------------------------------------------------------------------------------------
%	INTERESTS SECTION
%----------------------------------------------------------------------------------------

\section{interests}

\textbf{Computing:} Data analysis and visualisation, parallel and distributed computing, game design and programming, high-performance GPU computing, artificial intelligence, genetic algorithms, optimisation, Monte-Carlo simulations, 3D graphics and animation, real-time electric guitar amplifier synthesis, robotics, typesetting, web annotation, the semantic web.

\textbf{Academic:} Astrophysics of variable stars, astronomical observing and data reduction and analysis techniques, open-source educational software, programming literacy.

\textbf{Personal:} Guitars, DIY effects pedals and valve amplifiers, dragonboat racing, trail running, gaming.


%----------------------------------------------------------------------------------------
%	REFERENCES SECTION
%----------------------------------------------------------------------------------------

%\section{references}
%
%Dr Mark Horner\\
%Siyavula Education\\
%mark@siyavula.com\\
%\\
%Dr Carl Scheffler\\
%Siyavula Education\\
%carl@siyavula.com\\

\end{document}
